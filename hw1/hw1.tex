\title{CS 714 Assignment}
\author{Ruochen Lin}
\documentclass[11pt]{article}
\usepackage{amsmath,amsfonts,amssymb,amsthm}
\usepackage{mathtools}
\usepackage{commath}
\begin{document}
\maketitle
\pagebreak
\section{}
\subsection{}
The Taylor expansions centered at $x_2$ read
\begin{equation}\begin{split}
u(x_1) =& u(x_2) - u'(x_2)h_1 + \dfrac12 u''(x_2)h_1^2-\dfrac16u'''(x_2)h_1^3+\dfrac1{24}u^{(4)}(x_2)h_1^4+O(h_1^5)\\
u(x_3) =& u(x_2) + u'(x_2)h_2 + \dfrac12 u''(x_2)h_2^2+\dfrac16u'''(x_2)h_2^3+\dfrac1{24}u^{(4)}(x_2)h_2^4+O(h_2^5)\\
u(x_4) =& u(x_2) + u'(x_2)(h_2+h_3) + \dfrac12 u''(x_2)(h_2+h_3)^2+\dfrac16u'''(x_2)(h_2+h_3)^3 \\
	& +\dfrac1{24}u^{(4)}(x_2)(h_2+h_3)^4+O((h_2+h_3)^5) \\
\end{split}\end{equation} 
If we want to use function values at $\{x_i\}_{i=1}^4$ to approximate second derivative of $u$ at $x_2$, namely
$$u''(x_2) = au(x_1)+bu(x_2)+cu(x_3)+du(x_4),$$
we can pick $a-d$ by solving the following equations:
\begin{equation}\begin{split}
a+b+c+d = 0 \\
-h_1a+h_2c+(h_2+h_3)d = 0 \\
h_1^2a + h_2^2c+(h_2+h_3)^2d = 2 \\
-h1_3a+h_2^3c+(h_2+h_3)^3d = 0
\end{split}\end{equation} 
The solutions are
\begin{equation}\begin{split}
a &= \dfrac{2(2h_2+h_3)}{h_1(h_1+h_2)(h_1+h_2+h_3)} \\
b &= -a -c -d\\
c &= \dfrac{2(-h_1+h_2+h_3)}{h_2h_3(h_1+h_2)} \\
d &= \dfrac{2(h_1-h_2)}{h_3(h_2+h_3)(h_1+h_2+h_3)}
\end{split}\nonumber\end{equation} 

\subsection{}


\pagebreak
\section{}
\subsection{}
Suppose $u_0=u(0)$ and $u_{n+1} = u(10)$, and $f$ is indexed similarly, up to second order accuracy we have 
\begin{equation}\begin{split}
\dfrac1{h^2}(u_{k-1}-2u_k+u_{k+1}) + u_k = f_k,\ \ k=1,...,n
\end{split}\end{equation}
for points in the middle. For end points, using one-sided finite difference to approximate first-order derivative, we have 
\begin{equation}\begin{split}
u'(0) = -\dfrac3{2h}u_0+\dfrac2h u_1 - \dfrac1{2h}u_2 + O(h^2),
u'(10) = \dfrac1{2h}u_{n-1} -\dfrac2h u_n +\dfrac3{2h}u_{n+1} + O(h^2).
\end{split}\end{equation} 
Plug these expressions to the boundary condition, we have the following matrix system:
\begin{equation}\begin{split}
A &=\dfrac1{h^2}
\begin{bmatrix}
-\dfrac{3h}2-h^2 & 2h & -\dfrac h2 \\
1 & -2 + h^2 & 1 \\
& 1& -2 + h^2 & 1 \\
& & \ddots & \ddots & \ddots \\
& & & 1 & -2 + h^2 & 1 \\
& & & \dfrac{h}2 & -2h & \dfrac{3h}2 + h^2
\end{bmatrix} \\
u =& 
\begin{bmatrix}
u_0 \\ u_1 \\ \vdots \\ u_{n+1}
\end{bmatrix} 
f = 
\begin{bmatrix}
0 \\ f_1 \\ f_2 \\ \vdots \\ f_n \\ 0
\end{bmatrix} 
\end{split}\end{equation}
and $Au=f$.
\subsection{}
The general solution to the differential equation, without considering boundary conditions, is a particular solution to 
\begin{equation}
u''+u=-e^x
\end{equation} 
added to the gereral solution of 
\begin{equation}
v''+v = 0
\end{equation} 
Any solution to the inhomogeneous ODE can be written in this form because the difference between any two of them must be a solution to the homogeneous ODE. $u(x)=-\dfrac12e^x$ solves the inhomogeous equation, and the general solution to $v''+v=0$ is 
\begin{equation}\begin{split}
v = A\sin x + B\cos x.
\end{split}\nonumber\end{equation} 
Plugging in the boundary conditions, we get the values of $A$ and $B$:
$$A =B=\dfrac{e^{10}}{2\cos 10}.$$
Thus the exact solution is 
\begin{equation}\begin{split}
u(x) = -\dfrac12e^x+\dfrac{e^{10}}{2\cos 10}\sin x +\dfrac{e^{10}}{2\cos 10}\cos x. 
\end{split}\end{equation} 

\pagebreak

\section{}
\subsection{}
With
\begin{equation}\begin{split}
u(x\pm h) = u(x) \pm hu'(x) + \dfrac{h^2}2u''(x) \pm\dfrac{h^3}6 u^{(3)}+\dfrac{h^4}{24}u^{(4)}\pm\dfrac{h^5}{120}u^{(5)}+O(h^6),\\
u(x\pm 2h) = u(x) \pm 2hu'(x) + {2h^2}u''(x) \pm\dfrac{4h^3}3 u^{(3)}+\dfrac{2h^4}{3}u^{(4)}\pm\dfrac{4h^5}{15}u^{(5)}+O(h^6),
\end{split}\end{equation} 
we get the following expression for second-order derivatives for $k=0,...,n$:
\begin{equation}\begin{split}
u''(kh) = -\dfrac1{12h^2}(u_{k-2}+u_{k+2}) + \dfrac4{3h^2}(u_{k-1}+u_{k+1})+\dfrac5{2h^2}u_k + O(h^4).
\end{split}\end{equation}
Utilizing the periodicity of the problem, we have $u_0 = u_{n+1}$, $u_{-1}=u_{n}$ and $u_{n+2} = u_1$, so that the expression makes sense for all $k$s. Thus we have

\subsection{}
Similar to 2(b), we can assemble the exact solution by finding a particular solution to the inhomogeeous equation first, and then use the boundary condition to find the appropriate solution to the homogeneous equation to add to it. It turns out that \\
\begin{equation}\begin{split}
u(x) = \dfrac{\sin 4\pi x}{16\pi^2 + 1}
\end{split}\end{equation}
is the exact solution to the problem, because if the coefficients of $e^x$ or $e^{-x}$ are non-zero, the periodic boundary condition will be violated.
\end{document}

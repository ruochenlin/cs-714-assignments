\title{CS 714 Assignment 3}
\author{Ruochen Lin}
\documentclass[11pt]{article}
\usepackage{amsmath,amsfonts,amssymb,amsthm}
\usepackage{mathtools}
\usepackage{commath}
\begin{document}
\maketitle
\pagebreak

\section{}
\subsection{}
In Adams-Moulton methods, we get
\begin{equation}
    \sum_{j=0}^r \alpha_j = 0
\end{equation}
for free, and in the 2-step scheme we have
\begin{equation}
\alpha_0 = 0,\,\alpha_1 = -1,\,\alpha_2 = 1.
\nonumber\end{equation}
We can pick $\{\beta_j\}_{j=0}^2$ such that
\begin{equation}\begin{split}
    &\sum_{j=0}^2\beta_j = \alpha_1 + 2\alpha_2 = 1,\\
    &\sum_{j=0}^2 j\beta_j = \dfrac12\alpha_1+2\alpha_2 = \dfrac32,\\
    &3\sum_{j=0}^2 j^2\beta_j = \alpha_1+ 8\alpha_2 = 7,
\end{split}\end{equation}
which leads to
\begin{equation}
    \beta_0 = -\dfrac1{12},\,\beta_1 = \dfrac23,\,\beta_2 = \dfrac5{12}
\nonumber\end{equation}
so that
\begin{equation}
    U^{n+2} = U^{n+1} + k\Big[-\dfrac1{12} f(U^n) + \dfrac23f(U^{n+1})
        + \dfrac5{12}f(U^{n+2})\Big].
\end{equation}


\subsection{}
If we interpolate $f(u(t))$ with a quadratic function
\begin{equation}
    f(u(t)) = a(t-t_n)^2 + b(t-t_n) + c
\end{equation} 
with $f(u(t_n))=f(U^n)$, $f(u((t_{n+1}))) = f(U^{n+1})$, and 
$f(u((t_{n+2})) = f(U^{n+2})$, we have:
\begin{equation}\begin{split}
    f(U^n) =& c,\\
    f(U^{n+1}) =& ah^2 + bh + c,\\
    f(U^{n+2}) =& 4ah^2 + 2bh + c,
\end{split}\end{equation}
solving which leads to
\begin{equation}\begin{split}
    a &= \dfrac{f(U^{n+2}) - 2f(U^{n+1}) + f(U^n)}{2h^2},\\
    b &= \dfrac{-f(U^{n+2}) + 4f(U^{n+1}) + -3f(U^n)}{2h},\\
    c &= f(U^n).
\end{split}\end{equation} 
Plug this interpolated $f$ into the Newton-Leibniz equation for $u$, we have:
\begin{equation}\begin{split}
    U^{n+2}=&U^{n+1}+\int_{t_n+1}^{t_{n+2}}f(u(s))ds\\
           =&U^{n+1}+\int_h^{2h}(a(t-t_n)^2+b(t-t_n)+c)d(t-t_n)\\
           =&k\Big[-\dfrac1{12}f(U^{n}) + \dfrac23f(U^{n+1}) + \dfrac5{12}f(U^{n+2})\Big],
\end{split}\end{equation}
which is exactly the same as what we got from approach 1.

\pagebreak
\section{}
\subsection{}
Solving
\begin{equation}
    \xi^3+2\xi^2-4\xi-8 = 0
\end{equation} 
gives us $\xi_1=\xi_2=-2$, $\xi_3=2$. Thus the general solution would be
\begin{equation}
    U^n = (-2)^na + (-2)^nnb + 2^nc,\,\,\,a, b,c \in \mathbb{C}.
\end{equation} 
\subsection{}
Plug in the values of $\{U^n\}_{n=0}^2$ and solve for $a$, $b$, and $c$, we have:
\begin{equation}
    a=3,\,b=-1,\,c=1.
\nonumber\end{equation} 
The particular solution would be
\begin{equation}\begin{split}
    U^n = 3(-2)^n -n (-2)^n +2^n\\
\end{split}\end{equation} 

\pagebreak
\section{}

\pagebreak
\section{}
\subsection{}

\subsection{}

\subsection{}
Plug
\begin{equation}
    f(t, u) = \lambda(t)u,
\nonumber\end{equation}
into the Midpoint Method, we have:
\begin{equation}
    U^{n+1} = U^n + k\lambda(t_n + \dfrac k2)\cdot\dfrac{U^n+U^{n+1}}2,
\nonumber\end{equation}
or
\begin{equation}
    \abs{\dfrac{U^{n+1}}{U^n}} =
    \abs{\dfrac{1+\dfrac{k\lambda(t_n+\dfrac k2)}2}{1-\dfrac{k\lambda(t_n+\dfrac k2)}2}}.
\end{equation}
If $k>0$ and $\operatorname{Re}(\lambda(t)) \leq 0$, we have:
\begin{equation}\begin{split}
    \abs{\operatorname{Re}\Big(1+\dfrac{k\lambda(t_n+\dfrac k2)}2\Big)}
    &\leq\abs{\operatorname{Re}\Big(1-\dfrac{k\lambda(t_n+\dfrac k2)}2\Big)},\\
    \abs{\operatorname{Im}\Big(1+\dfrac{k\lambda(t_n+\dfrac k2)}2\Big)}
    &=\abs{\operatorname{Im}\Big(1-\dfrac{k\lambda(t_n+\dfrac k2)}2\Big)}.
\end{split}\end{equation}
Thus we have
\begin{equation}
    \abs{\dfrac{U^{n+1}}{U^n}} \leq 1
\end{equation}
by Pythagorean Theorem.

\subsection{}
In the Trapezoidal scheme, we would have 
\begin{equation}\begin{split}
    U^{n+1} &= U^n + \dfrac k2 \Big[ \lambda(t_nU^n + \lambda(t_{n+1})U^{n+1})\Big]\\
    \Longrightarrow
    \abs{\dfrac{U^{n+1}}{U^n}} &= \abs{\dfrac{1+\dfrac k2\lambda(t_n)}
        {1-\dfrac k2 \lambda(t_{n+1})}}.
\end{split}\end{equation}
Since $\lambda(t)$ can be very different at $t_n$ and $t_{n+1}$, 
the inequality in \textbf{4.3} is not guaranteed, and thus Trapezoidal Method is not AN-stable.
\end{document}

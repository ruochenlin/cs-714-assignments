\title{CS 714 Assignment 3}
\author{Ruochen Lin}
\documentclass[11pt]{article}
\usepackage{amsmath,amsfonts,amssymb,amsthm}
\usepackage{mathtools}
\usepackage{commath}
\begin{document}
\maketitle
\pagebreak

\section{}
\subsection{}

\subsection{}
If we interpolate $f(u(t))$ with a quadratic function
\begin{equation}
    f(u(t)) = a(t-t_n)^2 + b(t-t_n) + c
\end{equation} 
with $f(u(t_n))=f(U^n)$, $f(u((t_{n+1}))) = f(U^{n+1})$, and 
$f(u((t_{n+2})) = f(U^{n+2})$, we have:
\begin{equation}\begin{split}
    f(U^n) =& c \\
    f(U^{n+1}) =& ah^2 + bh + c \\
    f(U^{n+2}) =& 4ah^2 + 2b^h + c,
\end{split}\end{equation}
solving which leads to
\begin{equation}\begin{split}
    a &= \dfrac{f(U^{n+2}) - 2f(U^{n+1}) + f(U^n)}{2h^2}\\
    b &= \dfrac{-f(U^{n+2}) + 4f(U^{n+1}) + -3f(U^n)}{2h}\\
    c &= f(U^n)
\end{split}\end{equation} 
Plug this interpolated $f$ into the Newton-Leibniz equation for $u$, we have:
\begin{equation}\begin{split}
    U^{n+2}=&U^{n+1}+\int_{t_n+1}^{t_{n+2}}f(u(s))ds\\
           =&U^{n+1}+\int_h^{2h}(a(t-t_n)^2+b(t-t_n)+c)d(t-t_n)\\
           =&-\dfrac1{12}f(U^{n}) + \dfrac23f(U^{n+1}) + \dfrac5{12}f(U^{n+2}),
\end{split}\end{equation}
which is exactly the same as what we got from approach 1.

\pagebreak
\section{}

\pagebreak
\section{}

\pagebreak
\section{}
\subsection{}

\subsection{}

\subsection{}
Plug
\begin{equation}
    f(t, u) = \lambda(t)u,
\nonumber\end{equation}
into the Midpoint Method, we have:
\begin{equation}
    U^{n+1} = U^n + k\lambda(t_n + \dfrac k2)\cdot\dfrac{U^n+U^{n+1}}2,
\nonumber\end{equation}
or
\begin{equation}
    \abs{\dfrac{U^{n+1}}{U^n}} =
    \abs{\dfrac{1+\dfrac{k\lambda(t_n+\dfrac k2)}2}{1-\dfrac{k\lambda(t_n+\dfrac k2)}2}}.
\end{equation}
If $k>0$ and $\operatorname{Re}(\lambda(t)) \leq 0$, we have:
\begin{equation}\begin{split}
    \abs{\operatorname{Re}\Big(1+\dfrac{k\lambda(t_n+\dfrac k2)}2\Big)}
    &\leq\abs{\operatorname{Re}\Big(1-\dfrac{k\lambda(t_n+\dfrac k2)}2\Big)},\\
    \abs{\operatorname{Im}\Big(1+\dfrac{k\lambda(t_n+\dfrac k2)}2\Big)}
    &=\abs{\operatorname{Im}\Big(1-\dfrac{k\lambda(t_n+\dfrac k2)}2\Big)}.
\end{split}\end{equation}
Thus we have
\begin{equation}
    \abs{\dfrac{U^{n+1}}{U^n}} \leq 1
\end{equation}
by Pythagorean Theorem.

\subsection{}
In the Trapezoidal scheme, we would have 
\begin{equation}\begin{split}
    U^{n+1} &= U^n = \dfrac k2 \Big[ \lambda(t_nU^n + \lambda(t_{n+1})U^{n+1})\Big]\\
    \Longrightarrow
    \abs{\dfrac{U^{n+1}}{U^n}} &= \abs{\dfrac{1+\dfrac k2\lambda{t_n}}
        {1-\dfrac k2 \lambda(t_{n+1})}}.
\end{split}\end{equation}
Since $\lambda(t)$ can be very different at $t_n$ and $t_{n+1}$, 
the inequality in \textbf{4.3} is not guaranteed, and thus Trapezoidal Method is not AN-stable.
\end{document}
